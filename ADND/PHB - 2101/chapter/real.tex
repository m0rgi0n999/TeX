\documentclass[../main.tex]{subfiles}

\fancyhead{}
\fancyhead[RO,LE]{\textcolor{white}{\changefont{The Real Basics}}}


\begin{document}
	\begin{multicols}{3}
		This section is intended for novice role-
		players. If you have enjoyed role--playing
		games before, don't be surprised if what you
		read here sonds familiar.
		Games come in a wide assortment of
		types board games, card games, word
		games, picture games, miniatures games.
		Even within thest: categories are subcatego-
		ries. Board gam~, for example. can be
		divided into path games, real estate games,
		military simulation games. abstract strategy
		games, mystery game, ilnd a host of others.
		Still, in all this mass of games. rolr--
		playing games are unique. They form iI cat·
		egory all their own Ih.. , dOC!Sn', overlap iIIny
		other category.
		For that reason, role-playing games ilre
		hard to dncribe. Comparisons don', work
		because there isn't anything similar to com-
		pare them to. At least , nol without streich·
		ing your imagi nat ion well beyond ils
		no rmal, everydiilY e:dension .
		But then. stretching your imagination is
		what rol~playing is all about . So In's try an
		analogy.
		Ifn4IIgine that you are pl.. ying .. simple
		board &<lime, called Snakes and Ladders.
		Your gO<lI is to gd from the boltom to the
		top of the board before all the other playe.rs.
		Along the w ..y .. re traps that can send you
		sliding b ..ck toward your starting position .
		There are also ladders that can let you jump
		ahead, closer to the finish space. So fou, it's
		pretty simple and prnty Siandard.
		Now let's chan~ a few things. Instead of
		a flat , fealurelHS board with a palh winding
		from side to side, let's have a maze. You are
		standing attheentrana, and you know Ihal
		Ihere's an exit somewhere, bUI you don't
		know where, You have 10 find It .
		Instead of snakes and ladden, we'll put in
		hidden doors and secret passages. Oon'l roll
		a die 10 see how far you move; you c.. n
		move as rar as you w .. nt . Move down the
		corridor 10 the inlers«tion. You can tum
		righ t, or lefl, or go str.. ight .. head, or go
		back the way you came. Or, as long as
		you're here, you can look for a hidden door.
		If you find one, It will open into anolher
		streich of conidor. Th.. t conidor mighllake
		you straighl to lhe exit or lead you inlo a
		blind alley. The only way to find oul is 10
		slep in and slart walking.
		Of course, given enough time, evenlually
		you'll find the exil . To k~ lhe game inter-
		esting. kt's put some other Ihings in the
		maze with you . Nasty things . Things like
		vilmpire b..ts and hobgoblins and zombies
		ilnd ogres. Of course, we'll give you a sword
		and a shield, so if you meel one of lhese
		things you can defend yourself . You do
		know how 10 use iI sword, don'l youl
		And there are other players in the maze as
		well. They have swords and shields, too.
		How do you suppose anolher player would
		react if you chance 10 meet1 He might
		attack, but he also might offer to team up.
		After all, even an ogre mighl think twia
		about attacking Iwo people c.. nying sharp
		swo rds and stout shields.
		Finally, lei's put the board somewhere
		you can'l see it. Let's give it 10 one of the
		playe:n and make thai player the referee.
		Inslead of looking at the board , you listen to
		the refelft ilS he describes whilt you can see
		from your position on the bO<lrd . You tell
		the referee whal you want 10 do and he
		moves your piece accordingly, As the ref-
		ereoe describes your surroundings, try to pic-
		ture them mentally. Close your eyes and
		conslruct the Willis of lhe maze .. round
		yourself. Imagine the hobgoblin iI$ the ref·
		eree describes it whooping and gamboling
		down the conidor toward you. Now imag-
		ine how you would react in Ihill siluation
		and tellihe referee whal you are going to do
		about it.
		We have jusl constructed a simple role·
		playinggilme. It is not a sophisticated &<lime,
		but it hilS the essenlial element that makes iI
		rol~playing game; The player is placed in
		the midst of an unknown or dangerous situ-
		ation cre.. ted by a referee and must work his
		way Ihrough it .
		This is the heart of role-playing. The
		player adopts the role of a character and
		then guides that character Ihrough an
		adventure, The player makes decisions,
		inleracts with olher characters and players,
		and, essentially, ~pretmds~ 10 be his charac-
		ler during the course of the game. That
		doesn't mean that the player must jump up
		and down, dash around, and act like his
		character. It means that whenever the char-
		acter is called on 10 do somelhing or make a
		decision, the player pretends that he is in
		Ihat situation and chooses an appropriate
		course of action .
		Physically, the playen and referee (the
		OM) should be seated comfortably around a
		table w ith the referee at the head . Players
		need p lenty of room for papers, pencils,
		dice, rule books, drinks, and snacks, The
		referee needs extra space for his maps, dice,
		rule books, and asso rted notes.
		The Goal
		Another major dif-
		fenma between rol~playing games and
		other games is Ihe ultimate gO<lI. Everyone
		assumes Ihal a game must ~ve a beginning
		and an end and that the end comes when
		someone wins. That doesn't apply to role-
		playing because no one ~wins~ in a rol~
		playing game. The point of playing is not 10
		win but to have fun and to 5OCi.. lize.
		An adventure usually has a gO<lI of some
		sorl: proted the villagers from Ihe mon-
		sten; rescue the lost princess; explore the
		ancient ruins. Typically, this goal can be
		attained in a reasonable playing time: four
		to eight hours is standard . This might
		require the players to get together for one,
		9
		two, or even Ihree playing sessions to reach
		lheir goal ilnd complete the adventure.
		But the game doesn't end when an adven-
		tu re is finished . The same characters un go
		on to new adventures, Such a series of
		advenlures is called a campaign .
		Remember, the point of an .. dventure is
		not to win but to have fun while working
		toward a common goal. But the length of
		any particular adventure need not impose
		iIIn ilrtifici ..llimil on lhe length of the game .
		The AD\&D\textsuperscript{\textregistered} game embraces more than
		enough adventure to keep a group of char-
		acters occupied for years.
		Required Materials
		Aside from a copy
		of this book, very litt le is needed to play the
		AD\&Ogame.
		You will need some sort of character
		record, TSR publishes character record
		sheets Ih .. t are quite handy and easy to use,
		but any sheoet of paper will do . Blank pilper,
		lined paper, or even graph paper can be
		used. A double-sized sheet of paper (1 1 x 17
		inches), fo lded in half. is ex«1lent . Keep
		your character record in pencil, because it
		will change tTequently during the game, A
		good eraser is a.Iso a must .
		A full set o f polyhedral dia is necessary.
		A full set consisls of 4-, 6-, 8-, to.., U-, and
		2()..slded dia. A few exira 6- and l()..sided
		dice are a good idea. Polyhedral dia should
		be available wherever you got Ihis book.
		Througho ut Ihese rules,. the various dJce
		are referred 10 by a code thai is in the fonn:
		, of dice, follo wed by "d.." foUowed by a
		numeral fo r the type of dice, In o ther
		wo rds, If yo u Ire to roUo ne6-slded die, yo u
		wo uld see "roll ld6;'" Five ll-sided d ice are
		referred to .. N5 d 12," (If yo u do n' t have fi ve
		12" lded d i~, lust roU one five times and
		add the result ...)
		When the rules say 10 roll "pt!f"C'entUe
		dice" o r "ldtOO; yo u need to generate a
		rando m number fro m t to 100. One way to
		do this is to roll two l().sjded dice of dlffe ....
		ent colors. Before yo u roIL de:s.igniilte O M
		die .. the tens place and the other as the
		o nes place. RoWng them together enlbles
		yo u to genera Ie a nwnber from 1 to 100 (a
		result o f "0" o n both dJee is read iiI.S "00" o r
		"100"). Fo r example, if the blue cUe (repre-
		RntJnB the tens place) rolls an "8" and ~
		red dJe (ones place) rolls a "5," the result Is
		85. Ano ther, more expensive, w ay to genu ·
		ate a number fro m I to 100 is to buy one 01
		Ihe dJce Ihlt actually have num bers fro m 1
		10 100 o n them.
		At least one player should have .. few
		sheets of graph paper fo r m .. pping Ihe
		group's progress. Assorted pieces of SCTatch
		paper are handy for making quick notes, fo r
		passing secret messages to other playen or
		the OM, or for keeping track of odd bit" of
		infonnalion that you don't want cluttering
		The Real Basics
		up your char~cter rKord.
		Miniature figures are handy for keeping
		lUck of wht're everyone is in a confusing sit·
		uaUon like a battle. Th~ un be as elabo-
		rate or simple as you like. Some players use
		miniature lead or pewter figurts painted 10
		ruemble their characters. Plastic soldiers,
		chess piK'eS, bo.ardgame pawns, di«, or
		bits of paper n n work just as w,,11.
		An Example of Play
		To fu rther clarify
		what really goes on during an AD\&:O-
		game, read the following example. This is
		typicoilll of the sort of action thai occurs dur-
		ing a playing 56Sion .
		Shortly before this example begins. InTH;
		p!;IIyer charilcters fought a skirmish wit h a
		wererat (a ~alUre similar to a werewolf
		but which becomes an ~ormous rOIl instead
		of a wolf). The wereTal W~ wounded and
		fled down a tunnel. The characters are in
		punuil. 1M group includn two fishier'S
		ilnd a cleric. Flghter 1 is the group's leader.
		OM: You've been following this tunnel for
		about 120 yards. The water on the floor
		is ankle deep ilnd very cold. Now and
		then you feel something brush against
		your fool. The smell of decay Is getting
		stronger. The tunnel is gradually filling
		with a cold mist.
		Fighter 1; I don't like this at all . Can ~ see
		anything up ahead that loolulike iI door-
		way, or a branch in the tunnel7
		OM; Within the range of your torchlight,
		the tunnel is mo~ or less straight . You
		don't see any branches or doorways.
		Cleric: The we~r.1It we hit had 10 come thi5
		way. There's nowhere else to go.
		Fighter 1: Unless we missed a hidden door
		alol\8 the way. I hale this place; It gives
		me the creeps.
		Fighter 2: We have to trilck down thOilt were--
		rat . I uy we keep going.
		Fighter 1: OK, We keep moving down the
		tunneJ . But keep your eyes o~ for any-
		thing that might ~ iI door.
		OM : Another 30 or 35 yards down the tun-
		nel. you find iI stone block on the floor.
		Fighter 1: A block7 I take a clO5t'r look.
		OM ; It's a cu t block, about 12 by 16 inches,
		and 18 inches or so high. II looks like a
		different kind of rock than the rest of the
		tun nel.
		Fighter 2: Where is ill Is it in the cenler of
		the tunnel or off 10 the side?
		OM : 1t'5 right up agaWl tM side.
		Fighter 1; Can I move it7
		OM (checking the character's Strength
		score): Yeah, you can push it around
		without too much trouble.
		Fighter 1: Hmmm . This is obviously a
		marker of lOme sort. I want to check 1:'\5
		artil for secret doors. Spread oul and
		e)(amlne the walls,
		OM (rolls several dice behind his rule book,
		where players can't see the rnuhs) :
		Nobody finds anything unusUilI along
		the walls.
		Fighter 1: It has to br: here somewhere.
		What ilbout the ceiling?
		OM: You can't reach the ceiling. 1t'5 about a
		foot ~yond you r reach.
		Cleric: Of courser That block isn't a ma rker,
		it's a step. I climb up on the block and
		start prodding the ailing.
		OM (rolling a few more dice): You poke
		around for 20 seconds or so, then sud-
		denly part of the tunnel roof shifts .
		You've found a panel that lifu away.
		Fighter 1 : Open il very carefully.
		C leric: I pop it up a few inches and push il
		aside slowly. Can I see anything?
		OM: Your head is still below the level of the
		opening, but you see some dim light
		from one side.
		Fighter 1 : We boost him up so he can get a
		beller look.
		OM : OK. your friends boost you up into the
		room ...
		Fighter 1: No, nor We boost him just high
		enough to get his head through the o~­
		ing.
		OM : OK, you boost him up a foot. The two
		of you are each holding one of his legs.
		Cleric, you see another tunnel , pretty
		much like the one you were in, bUI il
		only goes off in one dir«tion . There's a
		doorway about 10 yards away with a
		soft light inside. A line of muddy
		pawprinls Iuds (rom tM hole you're in
		10 the doorway.
		aeric: Fine. I wantlhe fighters 10 go first.
		OM: & they're lowering you back 10 the
		block , everyone hears some grunls,
		splashing, and clanking weapons coming
		from furth~r down Ihe lower tunnel.
		They seem to ~ closing fa5t.
		aeric: upr Up! Push me back up Ihrough
		the hole! I grab the ledge and haul myself
		up. I'll help pull the next guy up.
		(All tiu'ee characters scramble up through
		the hole.)
		OM ; What about the piIInel7
		Fighter 1: We push it back into place.
		OM, It slides back into its slot wi th a nice,
		loud Hclunk .'" The grunting from below
		gets a lo t louder.
		Fighler 1: Creat, Ihey heard it. Cleric, get
		over here and sland on this panel. We're
		going to check out Ihat doorway.
		OM: Cleric, you hear some shouting and
		shuffling around below you, then there's
		a thump and the pilnel you're standing
		on lurches.
		Cleric; They're trying to batter il open!
		OM (to Ihe fighters): When you peN around
		lhe doorway. you see a small, dirty room
		with a smaJl col , a table, and a couple of
		stools. On the col is a wererat curled up
		into a ball , Its bOllck is toward you.
		There's anothe r door in the far wall and a
		10
		small gong in the comer.
		Fighter 1: Is the werera! moving7
		OM : Not a bit. Cleric, the panel lust
		thumped again. You can see a little crack
		In it now.
		Cleric: 00 something quick. you guys.
		When this panel starts coming apart, I'm
		gelling off it.
		Fighler 1: OK already I I step into Ihe room
		ilnd prod the wereut with my shield .
		What happens?
		OM: Nothing. You see blood on the cot .
		Fighter 1: Is this the same wererat we fought
		before7
		OM: Who knows? All wereral5 look the
		same to you. Cleric. the panel thumps
		again . That crack is looking really big.
		Cleric: That's it . I get off the panel. " m mov-
		ing into the room wit h ev~rybody else.
		OM : There's a ttm1endous smash and you
		hea r chunks of rock banging around out
		in the corridor. followed by lots of snarl-
		ing and squeaking. You Ieot flashes of
		lorchlight and werer.1ll shadows through
		1M doorway.
		Fighter 1: All righi , the other fighler and I
		move up 10 block the doorway, Thai's
		the narrowest area, they can onl y come
		Ihrough it one or two at a time. Cleric,
		you ,tily in the room and be ready with
		your spells.
		Fighter 2: At last, ill decent, stand-up fight!
		OM : As the first wererat appeiln in the
		doorway with a spear in his paws, you
		he.Jr a slam behind you.
		Cleric: I spin around. Whilt is it?
		OM: The door in the b.ack of the room is
		broken olf its hinges. Standing in the
		doorway, holding a milce in each P.JW, is
		the biggest, ugliest wererat you've ever
		seen, A couple more pairs of red eyes are
		shining through the darkness behind
		him . He's licking his chops in a way that
		you rind very unsettling.
		Cleric: Aaaaarrrghr I scream the name of
		my deity at the top of my lungs and then
		flip over the cot with the dead wererat on
		it so the body lands in front of him. I've
		got to have some help here, guys.
		Fighter 1 (to fighter 2); Help him, I'll handle
		this end of the room . (To OM): I'm
		attacking Ihe wererill in the doorway.
		OM : While fighter 2 is switching positions,
		the big wererat looks at the body on th~
		floor and his jaw drops. He loolu back
		up and says, 'That's Ignatz. He was my
		brother. You killed my brother.H Then he
		raises bolh maces ilnd leaps al you.
		At this point a ferocious melee breaks
		out. The OM uses the comb.Jt rules to play
		out the battle . If Ihe ch,Jir.Jcters survive, they
		can continue on whiltever course they
		choose.
	\end{multicols}
\end{document}
