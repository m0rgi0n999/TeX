\documentclass[../main.tex]{subfiles}

\fancyhead{}
\fancyhead[RO,LE]{\textcolor{white}{\changefont{Glossary}}}


\begin{document}
	\begin{multicols}{3}
\textbf{Ability} - any of the six natural traits that represent the basic definition of a player character: Strength, Dexterity, Constitution, Intelligence, Wisdom, and Charisma. A player character's abilities are determined at the beginning of a game by rolling 6-sided dice (d6s). The scores continue to be used throughout the game as a means of determining success or failure of many actions.
\textbf{Ability check} - a 1d20 roll against one of your character's ability scores (modifiers may be added to or subtracted from the die roll). A result that is equal to or less than your character's ability score indicates that the attempted action succeeds.
\textbf{AC} - abbreviation for Armor Class \textit{(q.v.)}.
\textbf{Alignment} - a factor in defining a player character that reflects his basic attitude toward society and the forces of the universe. Basically there are nine categories demonstrating the character's relationship to order vs. chaos and good vs. evil. A player character’s alignment is selected by the player when the character is created.
\textbf{Area of effect} - the area in which a magical spell or a breath weapon works on any creatures unless they make a saving throw.
\textbf{Armor Class} (\textit{abbr.} \textbf{AC}) - a rating for the protective value of a type of armor, figured from 10 (no armor at all) to 0 or even —10 (the best magical armor). The higher the AC, the more vulnerable the character is to attack.
\textbf{Attack roll} - The 1d20 roll used to determine if an attack is successful.
\textbf{Bend bars/lift gates roll} - the roll of percentile dice to determine whether a character succeeds in bending metal bars, lifting a heavy portcullis, or similar task. The result needed is a function of Strength and can be found in Table 1.
\textbf{Bonus spells} - extra spells at various spell levels that a priest is entitled to because of high Wisdom; shown in Table 5.
\textbf{Breath weapon} - the ability of a dragon or other creature to spew a substance out of its mouth just by breathing, without making an attack roll. Those in the area of effect must roll a saving throw.
\textbf{Cha} - abbreviation for Charisma (q. v.).
\textbf{Chance of spell failure} - the percentage chance that a priest spell will fail when cast. Based on Wisdom, it is shown in Table 5.
\textbf{Chance to know spell} - the percentage chance for a wizard to learn a new spell. Based on Intelligence, it is shown in Table 4.
\textbf{Charisma} (\textit{abbr.} \textbf{Cha}) - an ability score representing a character's persuasiveness, personal magnetism, and ability to lead.
\textbf{Common} - the language that all player characters in the AD&D® game world speak. Other languages may require the use of proficiency slots.
\textbf{Con} - abbreviation for Constitution \textit{(q.v.)}.
\textbf{Constitution} (\textit{abbr.} \textbf{Con}) - an ability score that represents a character's general physique, hardiness, and state of health.
\textbf{d} - abbreviation for dice or die. A roll that calls for.2d6, for example, means that the player rolls two six-sided dice.
\textbf{d3} - since there is no such thing as a threesided die, a roll calling for d3 means to use a d3, making 1 and 2 be a 1, 3 and 4 be a 2, and 5 and 6 be a 3.
\textbf{d4} -a four-sided die.
\textbf{d6} - a six-sided die.
\textbf{d8} - an eight-sided die.
\textbf{d10} - a ten-sided die. Two d10s can be used as percentile dice \textit{(q.v.)}.
\textbf{d12} - a twelve-sided die.
\textbf{d20} - a twenty-sided die.
\textbf{d100} - either an actual 100-sided die or two different-colored ten-sided dice to be rolled as percentile dice \textit{(q.v.)},
\textbf{DMG} -a reference to the \textit{Dungeon Master's Guide}.
\textbf{Damage} - the effect of a successful attack or other harmful situation, measured in hit points.
\textbf{Demihuman} -a player character who is not human: a dwarf, elf, gnome, half-elf, or halfling.
\textbf{Dex} - abbreviation for Dexterity \textit{(q.v.)}.
\textbf{Dexterity} (\textit{abbr.} \textbf{Dex}) - an ability score representing a combination of a character's agility, reflexes, hand-eye coordination, and the like.
\textbf{Dual-class character} - a human who switches character class after having already progressed several levels. Only humans can be dual-classed.
\textbf{Encumbrance} - the amount, in pounds, that a character is carrying. How much he can carry and how being encumbered affects his movement rate are based on Strength and are shown in Tables 47 and 48. Encumbrance is an optional rule.
\textbf{Energy drain} - the ability of a creature, especially undead, to drain energy in the form of class levels from a character, in addition to the normal loss of hit points.
\textbf{Experience points} (\textit{abbr.} \textbf{XP}) - points a character earns (determined by the Dungeon Master) for completing an adventure, for doing something related to his class particularly well, or for solving a major problem. Experience points are accumulated, enabling the character to rise in level in his class, as shown in Table 14 for warriors, Table 20 for wizards, Table 23 for priests, and Table 25 for rogues.
\textbf{Follower} - a non-player character who works for a character for money but is initially drawn to his reputation.
\textbf{Gaze attack} - the ability of a creature, such as a basilisk, to attack simply by making eye contact with the victim.
\textbf{Henchmen} - non-player characters who work for a character mainly out of loyalty and love of adventure. The number of henchmen a character can have is based on Charisma and is shown in Table 6. The DM and the player share control of the henchmen.
\textbf{Hireling} - non-player characters who work for a character just for money. Hirelings are completely under the control of the DM.
\textbf{Hit dice} - the dice rolled to determine a character's hit points. Up to a certain level, one or more new Hit Dice are rolled each time a character attains a new class level. A fighter, for example, has only one 10-sided Hit Die (1d10) at 1st level, but when he rises to the 2nd level, the player rolls a second d10, increasing the character's hit points.
\textbf{Hit points} - a number representing 1. how much damage a character can suffer before being killed, determined by Hit Dice \textit{(q.v.)}. The hit points lost to injury can usually be regained by rest or healing. 2. how much damage a specific attack does, determined by weapon or monster statistics, and subtracted from a player's total.
\textbf{Infravision} - the ability of certain character races or monsters to see in the dark. Infravision generally works up to 60 feet in the darkness.
\textbf{Initiative} - the right to attack first in a combat round, usually determined by the lowest roll of a 10-sided die. The initiative roll is eliminated if surprise \textit{(q.v.)} is achieved.
\textbf{Int} - abbreviation for Intelligence \textit{(q.v.)}.
\textbf{Intelligence} (\textit{abbr.} \textbf{Int}) - an ability score representing a character's memory, reasoning, and learning ability.
\textit{Italic type} - used primarily to indicate spells and magical items.
\textbf{Level} - any of several different game factors that are variable in degree, especially: 1. class level, a measure of the character's power, starting at the 1st level as a beginning adventurer and rising through the accumulation of experience points to the 20th level or higher. At each level attained, the character receives new powers. 2. spell level, a measure of the power of a magical spell. A magic-using character can use only those spells for which his class level qualifies him. Wizard spells come in nine levels (Table 21); priest spells in seven (Table 24).
\textbf{Loyalty base} - a bonus added to or a penalty subtracted from the probability that henchmen are going to stay around when the going gets tough. Based on the character's Charisma, it is shown in Table 6.
\textbf{M} - abbreviation for material component \textit{(q.v.)},
\textbf{Magical defense adjustment} - a bonus added to or a penalty subtracted from saving throws vs. spells that attack the mind. Based on Wisdom, it is shown in Table 5.
\textbf{Maneuverability class} - a ranking for flying creatures that reflects their ability to turn easily in aerial combat. Each class—from a top of A to a bottom rank of E—has specific statistical abilities in combat.
\textbf{Material component} (\textit{abbr.} \textbf{M}) - any specific item that musi be handled in some way during the casting of a magic spell.a: —
\textbf{Maximum press} - the most weight a character can pick up and raise over his head. It isa function of Strength and may be found in Table 1.
\textbf{Melee} - combat in which characters are fighting in direct contact, such as with swords, claws, or fists, as opposed to fighting with missile weapons or spells.
\textbf{Missile combat} - combat involving the use of weapons that shoot missiles or items that can be thrown. Because the combat is not “toe-to-toe,” the rules are slightly different than those for regular combat.
\textbf{Movement rate} - a number used in calculating how far and how fast a character can move in a round. This number is in units of \textit{10 yards} per round outdoors, but it represents \textit{10 feet} indoors. Thus, an MR of 6 is 60 yards per round in the wilderness, but only 60 feet per round in a dungeon.
\textbf{MR} - abbreviation for movement rate \textit{(q.v.)}.
\textbf{Multi-class character} - a demihuman who improves in two or more classes at the same time by dividing experience points between the different classes. Humans cannot be multi-classed.
\textbf{Mythos} (\textit{pl.} \textbf{mythoi}) - a complete body of belief particular to a certain time or place, including the pantheon of its gods.
\textbf{Neutrality} - a philosophical position, or alignment, of a character that is between belief in good or evil, order or chaos.
\textbf{Non-player character} - any character controlled by the DM instead of a player.
\textbf{NPC} - abbreviation for non-player character \textit{(q.v.)}.
\textbf{Open doors roll} - the roll of a 20-sided die to see if a character succeeds in opening a heavy or stuck door or performing a similar task, The die roll at which the character succeeds can be found in Table 1.
\textbf{Opposition school} - a schoo! of magic that is directly opposed to a specialist's school of choice, thus preventing him from learning spells from that school, as shown in Table 22.
\textbf{PC} - abbreviation for player character \textit{(q.v.)}.
\textbf{Percentage (or percent) chance} - a number between 1 and 100 used to represent the probability of something happening. If a character is given an X percentage chance of an event occurring, the player rolls percentile dice \textit{(q.v.)}.
\textbf{Percentile dice} - either a 100-sided die or two 10-sided dice used in rolling a percentage number. If 2d10 are used, they are of different colors, and one represents the tens digit while the other is the ones.
\textbf{Player character} (\textit{abbr.} \textbf{PC}) - the characters in a role-playing game who are under the control of the players.
\textbf{Poison save} - a bonus or a penalty to a saving throw vs. poison. Based on Constitution, it is shown in Table 3.
\textbf{Prime requisite} - the ability score that is most important to a character class; for example, Strength to a fighter.
\textbf{Proficiency} - a character's learned skill not defined by his class but which gives him a greater percentage chance to accomplish a specific type of task during an adventure. Weapon and nonweapon proficiency slots are acquired as the character rises in level, as shown in Table 34. The use of proficiencies in the game is optional.
\textbf{Proficiency check} - the roll of a 20-sided die to see if a character succeeds in doing a task by comparing the die roll to the character's relevant ability score plus or minus any modifiers shown in Table 37 (the modified die rol] must be equal to or less than the ability score for the action to succeed). q.v. - “which see,” or “turn to.”
\textbf{Race} - a player character's species: human, elf, dwarf, gnome, half- elf, or halfling. Race puts some limitations on the PC's class.
\textbf{Rate of fire} (\textit{abbr.} \textbf{ROF}) - number of times a missile-firing or thrown weapon can be shot in a round.
\textbf{Reaction adjustment} - a bonus added to or penalty subtracted from a die roll used in determining the success of a character's action. Such an adjustment is used especially in reference to surprise (shown on Table 2 as a function of Dexterity) and the reaction of other intelligence beings to a character (shown on Table 6 as a function of Charisma).
\textbf{Regeneration} -a special ability to heal faster than usual, based on an extraordinarily high Constitution, as shown in Table 3.
\textbf{Resistance} - the innate ability of a being to withstand attack, such as by magic. Gnomes, for example, have a magic resistance that adds bonuses to their saving throws against magic (Table 9).
\textbf{Resurrection survival} - the percentage chance a character has of being magically raised from death, Based on Constitution, it is shown in Table 3.
\textbf{Reversible} - of a magic spell, able to be cast “backwards,” so that the opposite of the usual effect is achieved.
\textbf{ROF} - abbreviation for rate of fire \textit{(q.v.)}.
\textbf{Round} - in combat, a segment of time approximately 1 minute long, during which a character can accomplish one basic action. Ten combat rounds equal one turn.
\textbf{S} - abbreviation for somatic component \textit{(q.v.)}.
\textbf{Saving throw} - a measure of a character's ability to resist (to “save vs.") special types of attacks, especially poison, paralyzation, magic, and breath weapons. Success is usually determined by the roll of 1d20.
\textbf{School of magic} - One of nine different categories of magic, based on the type of magical energy utilized. Wizards who concentrate their work on a single school are called specialists. The specific school of which a spell is a part is shown after the name of the spell in the spell section at the end of the book.
\textbf{Somatic component} (\textit{abbr.} \textbf{S}) - the gestures that a spellcaster must use to cast a specific spell. A bound wizard cannot cast a spell requiring somatic components.
\textbf{Specialist} - a wizard who concentrates on a specific school of magic \textit{(q.v.)}, as opposed to a mage, who studies all magic in general.
\textbf{Spell immunity} - protection that certain characters have against illusions or other specific spells, based on high Intelligence (Table 4) or Wisdom (Table 5).
\textbf{Sphere of influence} - any of sixteen categories of clerical spells to which a priest may have major access (he can eventually learn them all) or minor access (he can learn only the lower level spells). The relevant sphere of influence is shown as the first item in the list of characteristics in the priest spells.
\textbf{Str} - abbreviation for Strength \textit{(q.v.)}.
\textbf{Strength} (\textit{abbr.} \textbf{Str}) - an ability score representing a character's muscle power, endurance, and stamina.
\textbf{Surprise roll} - the roll of a ten-sided die by the Dungeon Master to determine if a character or group takes another by surprise. Successful surprise (a roll of 1, 2, or 3) cancels the roll for initiative on the first round of combat.
\textbf{System shock} - a percentage chance that a character survives major magical effects, such as being petrified. Based on Constitution, it is shown in Table 3.
\textbf{THACO} - an acronym for “To Hit Armor Class 0," the number that a character needs to roll in order to successfully hit a target with AC 0.
\textbf{To-hit roll} - another name for attack roll \textit{(q.v.)}.
\textbf{Turn} - in game time, approximately 10 minutes; used especially in figuring how long various magic spells may last. In combat, a turn consists of 10 rounds.
\textbf{Turn undead} - a magical ability of a cleric or paladin to turn away an undead creature, such as a skeleton or a vampire.
\textbf{V} - abbreviation for verbal component \textit{(q.v.)}.
\textbf{Verbal component} - specific words or sounds that must be uttered while casting a spell.
\textbf{Weapon speed} - an initiative modifier used in combat that accounts for the time required to get back into position to reuse a weapon.
\textbf{Wis} - abbreviation for Wisdom \textit{(q.v.)}.
\textbf{Wisdom}( \textit{abbr.} \textbf{Wis}) - an ability score representing a composite of a character's intuition, judgment, common sense, and will power.
\textbf{XP} - abbreviation for experience points \textit{(q.v.)}.
	\end{multicols}
\end{document}