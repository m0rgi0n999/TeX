\documentclass[../main.tex]{subfiles}

\fancyhead{}
\fancyhead[RO,LE]{\textcolor{white}{\changefont{Chapter 1: Player Character Ability Scores}}}


\begin{document}
	\begin{multicols}{3}
To venture into the worlds of the AD&D\textsuperscript{\textregistered} game, you first need to create a character. The character you create is your alter ego in the fantasy realms of this game, a makebelieve person who is under your control and through whom you vicariously explore the world the Dungeon Master (DM) has created.\\
\indent{Each character in the AD&D game has six abilities: Strength, Dexterity, Constitution, Intelligence, Wisdom, and Charisma. The first three abilities represent the physical nature of the character, while the second three quantify his mental and personality traits.}\\
\highlight{In various places throughout these rules, the following abbreviations are used for the ability names: Strength—Str; Dexterity—Dex; Constitution—Con; Inaes Wisdom—Wis; Charisma—Cha.}

\section*{Rolling Ability Scores}
Let's first see how to generate ability scores for your character, after which definitions of each ability will be given.
\indent{The six ability scores are determined randomly by rolling six-sided dice to obtain a score from 3 to 18. There are several methods for rolling up these scores.}\\
\begin{itemize}
\item \textbf{Method I:} Roll three six-sided dice (3d6); the total shown on the dice is your character’s Strength ability score. Repeat this for Dexterity, Constitution, Intelligence, Wisdom, and Charisma, in that order. This method gives a range of scores from 3 to 18, with most results in the 9 to 12 range. Only a few characters have high scores (15 and above), so you should treasure these characters.
\end{itemize}

\subsection*{Alternative Dice-Rolling Methods}
Method I creates characters whose ability scores are usually between 9 and 12. If you would rather play a character of truly heroic proportions, ask your DM if he allows players to use optional methods for rolling up characters. These optional methods are designed to produce above-average characters.
\begin{itemize}
\item \textbf{Method II:} Roll 3d6 twice, noting the total of each roll. Use whichever result you prefer for your character's Strength score. Repeat this for Dexterity, Constitution, Intelligence, Wisdom, and Charisma. This allows you to pick the best score from each pair, generally ensuring that your character does not have any really low ability scores (but low ability scores are not all that bad anyway!).
\item \textbf{Method III:} Roll 3d6 six times and jot down the total for each roll. Assign the scores to your character's six abilities however you want. This gives you the chance to custom-tailor your character, although you are not guaranteed high scores.
\item \textbf{Method IV:} Roll 3d6 twelve times and jot down all twelve totals. Choose six of these rolls (generally the six best rolls) and assign them to your character's abilities however you want. This combines the best of methods II and III, but takes somewhat longer. As an example, Joan rolls 3d6 twelve times and gets results of 12, 5, 6, 8, 10, 15, 9, 12, 6, 11, 10, and 7. She chooses the six best rolls (15, 12, 12, 11, 10, and 10) and then assigns them to her character's abilities so as to create the strengths and weaknesses that she wants her character to have (see the ability descriptions following this section for explanations of the abilities).
\item \textbf{Method V:} Roll four six-sided dice (4d6). Discard the lowest die and total the remaining three. Repeat this five more times, then assign the six numbers to the character's abilities however you want. This is a fast method that gives you a good character, but you can still get low scores (after all, you could roll 1s on all four dice!).
\item \textbf{Method VI:} This method can be used if you want to create a specific type of character. It does not guarantee that you will get the character you want, but it will improve your chances.
\indent{Each ability starts with a score of 8, Then roll seven dice. These dice can be added to your character's abilities as you wish. All the points on a die must be added to the same ability score. For example, if a 6 is rolled on one die, all 6 points must be assigned to one ability. You can add as many dice as you want to any ability, but no ability score can exceed 18 points. If you cannot make an 18 by exact count on the dice, you cannot have an 18 score.}
\end{itemize}

\section*{The Ability Scores}
The six character abilities are described below. Each description gives an idea of what that ability encompasses. Specific game effects are also given. At the end of each ability description is the table giving all modifiers and game information for each ability score. The unshaded area of these tables contains scores a player character can have naturally, without the aid of magical devices, spells, or divine intervention. The blue-shaded ability scores can be obtained only by extraordinary means, whether by good fortune (finding a magical book that raises a score) or il! fortune (an attack by a creature that lowers a score).

\subsection*{Strength}
Strength (Str) measures a character's muscle, endurance, and stamina. This ability is the prime requisite of warriors because they must be physically powerful in order to wear armor and wield heavy weapons. A fighter with a score of 16 or more in Strength gains a 10 percent bonus to the experience points he earns.
\indent{Furthermore, any warrior with a Strength score of 18 is entitled to roll percentile dice (see Glossary) to determine exceptional Strength; exceptional Strength improves the character's chance to hit an enemy, increases the damage. he causes with each hit, increases the weight the character is able to carry without a penalty for encumbrance (see below), and increases the character's ability to force open doors and similar portals.}
\indent{The rest of this section on Strength consists of explanations of the columns in Table 1. Refer to the table as you read.}
\indent{\textbf{Hit Probability} adjustments are added to or subtracted from the attack roll rolled on 1d20 (one 20-sided die) during combat. A bonus (positive number) makes the opponent easier to hit; a penalty (negative number) makes him harder to hit.}
\indent{\textbf{Damage Adjustment} also applies to combat. The listed number is added to or subtracted from the dice rolled to determine the damage caused by an attack (regardless of subtractions, a successful attack roll can never cause less than 1 point of damage). For example, a short sword normally causes 1d6 points of damage (a range of 1 to 6), An attacker with Strength 17 causes one extra point of damage, for a range of 2 to 7 points of damage. The damage adjustment also applies to missile weapons, although bows must be specially made to gain the bonus; crossbows never benefit from the user's Strength.}
\indent{\textbf{Weight Allowance} is the weight (in pounds) a character can carry without being encumbered (encumbrance measures how a character's possessions hamper his movement—see Glossary). These weights are expressed in pounds. A character carrying up to the listed weight can move his full movement rate.}
\indent{\textbf{Maximum Press} is the heaviest weight a character can pick up and lift over his head. A character cannot walk more than a few steps this way. No human or humanoid creature without exceptional Strength can lift more than twice his body weight over his head. In 1987, the world record for lifting a weight overhead in a single move was 465 pounds. A heroic fighter with Strength 18/00 (see Table 1) can lift up to 480 pounds the same way and he can hold it overhead for a longer time!Ability Scores (Dexterity)}
\indent{\textbf{Open Doors} indicates the character's chance to force open a heavy or stuck door.When a character tries to force a door open, roll 1d20. If the result is equal to or less than the listed number, the door opens. A character can keep trying to open a door until it finally opens, but each attempt takes time (exactly how much is up to the DM) and makes a lot of noise. Numbers in parentheses are the chances (on 1d20) to open a locked, barred, or magically held door, but only one attempt per door can ever be made. If it fails, no further attempts by that character can succeed.}
\indent{\textbf{Bend Bars/Lift Gates} states the character's percentage chance (rolled on percentile dice) to bend normal, soft iron bars, lift a vertical gate (portcullis), or perform a similar feat of enormous strength. When the character makes the attempt, roll percentile dice. If the number rolled is equal to or less than the number listed on Table 1, the character bends the bar or lifts the gate. If the attempt fails, the character can never succeed at that task. A character can, however, try to bend the bars on a gate that he couldn't lift, and vice versa.} 

% Please add the following required packages to your document preamble:
% \usepackage{graphicx}
\begin{table}[]
	\caption{Table 1: STRENGTH}
	\label{Table 1: STRENGTH}
	\resizebox{\columnwidth}{!}{%
	\begin{tabular}{cccccccc}
	\begin{tabular}[c]{@{}c@{}}Ability\\ Score\end{tabular} & \begin{tabular}[c]{@{}c@{}}Hit\\ Prob.\end{tabular} & \begin{tabular}[c]{@{}c@{}}Damage\\ Adj.\end{tabular} & \begin{tabular}[c]{@{}c@{}}Weight\\ Allow.\end{tabular} & \begin{tabular}[c]{@{}c@{}}Max.\\ Press\end{tabular} & \begin{tabular}[c]{@{}c@{}}Open\\ Doors\end{tabular} & \begin{tabular}[c]{@{}c@{}}Bend Bars/\\ Lift Gates\end{tabular} & Notes \\
	1 & -5 & -4 & 1 & 3 & 1 & 0\% &  \\
	2 & -3 & -2 & 1 & 5 & 1 & 0\% &  \\
	3 & -3 & -1 & 5 & 10 & 2 & 0\% &  \\
	4-5 & -2 & -1 & 10 & 25 & 3 & 0\% &  \\
	6-7 & -1 & None & 20 & 55 & 4 & 0\% &  \\
	8-9 & Normal & None & 35 & 90 & 5 & 1\% &  \\
	10-11 & Normal & None & 40 & 115 & 6 & 2\% &  \\
	12-13 & Normal & None & 45 & 140 & 7 & 4\% &  \\
	14-15 & Normal & None & 55 & 170 & 8 & 7\% &  \\
	16 & Normal & +1 & 70 & 195 & 9 & 10\% &  \\
	17 & +1 & +1 & 85 & 220 & 10 & 13\% &  \\
	18 & +1 & +2 & 110 & 255 & 11 & 16\% &  \\
	18/01-50 & +1 & +3 & 135 & 280 & 12 & 20\% &  \\
	18/51-75 & +2 & +3 & 160 & 305 & 13 & 25\% &  \\
	18/76-90 & +2 & +4 & 185 & 330 & 14 & 30\% &  \\
	18/91-99 & +2 & +5 & 235 & 380 & 15(3) & 35\% &  \\
	18/00 & +3 & 36 & 335 & 480 & 16(6) & 40\% &  \\
	19 & +3 & +7 & 485 & 640 & 16(8) & 50\% & Hill Giant \\
	20 & +3 & +8 & 535 & 700 & 17(10) & 60\% & Stone Giant \\
	21 & +4 & +9 & 635 & 810 & 17(12) & 70\% & Frost Giant \\
	22 & +4 & +10 & 785 & 970 & 18(14) & 80\% & Fire Giant \\
	23 & +5 & +11 & 935 & 1130 & 18(16) & 90\% & Cloud Giant \\
	24 & +6 & +12 & 1235 & 1440 & 19(17) & 95\% & Storm Giant \\
	25 & +7 & +14 & 1535 & 1750 & 19(18) & 99\% & Titan
	\end{tabular}%
	}
\end{table}

tions, beneficial Dexterity modifiers to Constitution
Dexterity
Dexterity (Dex) encompasses several physical attributes
including hand-eye coordination, agility,
reaction speed, reflexes, and balance. Dexterity affects a character's reaction to a
threat or surprise, his accuracy with thrown
weapons and bows, and his ability to dodge
an enemy's blows. It is the prime requisite of
rogues and affects their professional skills.
A rogue with a Dexterity score of 16 or higher gains a 10 percent bonus to the experience
points he earns.
Reaction Adjustment modifies the die roll
to see if a character is surprised when he
unexpectedly encounters NPCs. The more
positive the modifier, the less likely the
character is to be surprised,
Missile Attack Adjustment is used to
modify a character's die roll whenever he
uses a missile weapon (a bow or a thrown
weapon). A positive number makes it easier
for the character to hit with a missile, while
a negative number makes it harder.
Defensive Adjustment applies to a character’s saving throws (see Glossary) against
attacks that can be dodged—lightning bolts,
boulders, etc. It also modifies the character's
Armor Class (see Glossary), representing
his ability to dodge normal missiles and parry weapon thrusts. For example, Rath is
wearing chain mail, giving him an Armor
Class of 5. If his Dexterity score is 16, his
Armor Class is modified by —2 to 3, making him harder to hit. If his Dexterity score
is 5, his Armor Class is modified by +2 to 7,
making him easier to hit. (In some situaArmor Class do not apply. Usually this
occurs when a character is attacked from
behind or when his movement is
restricted—attacked while prone, tied up,
on a ledge, climbing a rope, etc.)
Table 2; DEXTERITY
Missile
Ability Reaction Attack Defensive
Score Adj. Adj. Adj.
1 -6 -6 +5
2 -4 -4 +5
3 =3 at +4
4 —2 m2 +3
5 = =a +2
6 O.— Pia eR
7 O- 0 Tope
8 sae o oO
9 0 0 0
10-14 0 0 0
15 0 0 J
16 CA ee Re ae
LY nee ee pear
13 +2 Meh coe oe
19 +3 +3 -4
20 +3 +3 —4
a1 +4 +4 =—5
22 +4 +4 5
23 +4 +4 =5
24 +5 +5 -6
25 +5 +5 =6
14
A character's Constitution (Con) score encompasses his physique, fitness, health, and _ physical
resistance to hardship, injury, and disease.
Since this ability affects the character's hit
points and chances of surviving such tremendous shocks as being physically
reshaped by magic or resurrected from
death, it is vitally important to all classes.
Some classes have minimum allowable
Constitution scores.
A character's initial Constitution score is
the absolute limit to the number of times the
character can be raised or resurrected from
death. Each such revival reduces the character's Constitution score by one. Magic can
restore a reduced Constitution score to its
original value or even higher, but this has no
effect on the number of times a character
can be revived fram death! Once the character has exhausted his original Constitution,
nothing short of divine intervention can
bring him back, and divine intervention is
reserved for only the bravest and mos?
faithful heroes!
For example, Rath’s Constitution score at
the start of his adventuring career is 12. He
can be revived from death 12 times. If he
dies a 13th time, he cannot be resurrected or
raised.
Hit Point Adjustment is added to or subtracted from each Hit Die rolled for the
character. However, no Hit Die ever yields
less than 1 hit point, regardless of modifications. If an adjustment would lower the
number rolled to 0 or less, consider the finalAbility Scores (Intelligence)
result to be 1, Always use the character's
current Constitution to determine hit point
bonuses and penalties.
Only warriors are entitled to a Constitution bonus of +3 or +4. Non-warrior characters who have Constitution scores of 17 or
18 receive only +2 per die.
The Constitution bonus ends when a
character reaches 10th level (9th for warriors and priests)—neither the Constitution
bonus nor Hit Dice are added to a character's hit points after he has passed this level
(see the character class descriptions that
start on page 25).
If a character's Constitution changes during the course of adventuring, his hit points
may be adjusted up or down to reflect the
change. The difference between the character’s current hit point bonus (if any) and the
new bonus is multiplied by the character's
level (up to 10) and added to or subtracted
from the character's total. If Delsenora’s
Constitution increased from 16 to 17, she
would gain 1 hit point for every level she
had, up to 10th level.
System Shock states the percentage
chance a character has to survive magical
effects that reshape or age his body: petrification (and reversing petrification), polymorph, magical aging, etc. It can also be
used to see if the character retains consciousness in particularly difficult situations. For example, an evil mage
polymorphs his dim-witted hireling into a
crow. The hireling, whose Constitution
score is 13, has an 85 percent chance to survive the change. Assuming he survives, he
must successfully roll for system shock
again when he is changed back to his original form or else he will die.
Resurrection Survival lists a character's
percentage chance to be successfully resurrected or raised from death by magic. The
player must roll the listed number or less on
percentile dice for the character to be
revived. If the dice roll fails, the character is
dead, regardless of how many times he has
previously been revived. Only divine intervention can bring such a character back
again.
Poison Save modifies the saving throw
vs. poison for humans, elves, gnomes, and
half-elves. Dwarves and halflings do not use
this adjustment, since they have special
resistances to poison attacks. The DM has
specific information on saving throws.
Regeneration enables those with specially endowed Constitutions (perhaps by a
wish or magical item) to heal at an advanced
rate, regenerating damage taken. The character heals 1 point of damage after the passage of the listed number of turns. However,
fire and acid damage (which are more extensive than normal wounds) cannot be regenerated in this manner. These injuries must
heal normally or be dealt with by magical
means,
Table 3: CONSTITUTION
Ability Hit Point System
Score Adjustment Shock
1 -3 25%
2 -2 30%
43 42 35% 40%
5 a 45%
6 -1 50%
Z 0 55%
8 0 60%
9 0 65%
10 u °0 70% 75%
a vbG 80%
13 0 85 %
14 0 88 %
15 +1 90%
16 17 +2(+3)" +2. 95% 97%
18 +2 (+4)* 99%
19 +2 (+5)" 99%
20 +2 (+5)** 99%
21 #2 (+6)"*" 99%
22 $2(+6)"" 99%
23 #2:(+6n"** 99%
24 +2 ( +7)** « 99%
25 20075 100%
Resurrection Poison
Survival Save Regeneration
30% 2 Nil
35% -] Nil
45% 40% 00 Nil Nil
50% 0 Nil
55% 0 Nil
60% 0 Nil
65% 0 Nil
70% 0 Nil
75% 0 Nil
80% 0 Nil
— 85%. 0 Nil
90% 0 Nil
92% 0 Nil
04% 0 Nil
96% 0 Nil
98% 0 Nil ©
100% 0 Nil
100% +] Nil
100% +1 1/6 turns
100% +2 1/5 turns
100% +2 1/4 turns
100% +3 1/3 turns
100% +3 1/2 turns
100 % +4 1/1 turn
* Parenthetical bonus applies to warriors only. All other classes receive maximum bonus
of +2 per die.
** All 1s rolled for Hit Dice are automatically considered 2s.
*** All 1s and 2s rolled for Hit Dice are automatically considered 3s.
**** All 1s, 2s, and 3s rolled for Hit Dice are automatically considered 4s.
Intelligence
Intelligence (Int)
represents a character's memory, reasoning,
and learning ability, including areas outside
those measured by the written word. Intelligence dictates the number of languages a
character can learn. Intelligence is the prime
requisite of wizards, who must have keen
minds to understand and memorize magical
spells. A wizard with an Intelligence score
of 16 or higher gains a 10 percent bonus to
experience points earned. The wizard’s
Intelligence dictates which spells he can
learn and the number of spells he can memorize at one time. Only those of the highest
Intelligence can comprehend the mighty
magic of 9th-level spells.
This ability gives only a general indication of a character's mental acuity. A semiintelligent character (Int 3 or 4) can speak
(with difficulty) and is apt to react instinctively and impulsively. He is not hopeless as
a player character (PC), but playing such a
character correctly is not easy. A character
with low Intelligence (Int 5-7) could also be
called dull-witted or slow. A very intelligent
person (Int 11 or 12) picks up new ideas
quickly and learns easily. A highly intelligent character (Int 13 or 14) is one who can
solve most problems without even trying
15
very hard. One with exceptional intelligence (Int 15 or 16) is noticeably above the
norm. A genius character is brilliant (Int 17
or 18). A character beyond genius is potentially more clever and more brilliant than
can possibly be imagined.
However, the true capabilities of a mind
lie not in numbers—1.Q.., Intelligence score,
or whatever. Many intelligent, even brilliant, people in the real world fail to apply
their minds creatively and usefully, thus
falling far below their own potential. Don't
rely too heavily on your character's Intelligence score; you must provide your character with the creativity and energy he
supposedly possesses!
Number of Languages lists the number of
additional languages the character can
speak beyond his native language. Every
character can speak his native language, no
matter what his Intelligence is. This knowledge extends only to speaking the language;
it does not include reading or writing. The
DM must decide if your character begins the
game already knowing these additional languages or if the number shows only how
many languages your character can possibly learn. The first choice will make communication easier, while the second
increases your opportunities for roleplaying (finding a tutor or creating a reasonUe melee Micali ead)
Table 4: INTELLIGENCE
Ability # of Spell Chanceto Max.fof Spell
Score Lang. Level Learn Spell Spells/Lvi Immunity
1 oO a - —
2 1 — = _ —
3 l = — — —
4 ] — _ _ —
5 1 = = = =
6 1 _ _ = =
7 ] = _
8 ] _ — - a
9 2 4th 35% 6 —
10 2 5th 40% 7 ~—
1] 2 Sth 45% 7 —
12 3 6th 50% 7 _
13 3 6th 55% 9 _
14 4 7th 60% a _
15 4 7th 65% 11 —_
16 5 8th 70% 1] —
17 6 8th 75% 14
18 7 oth 85% 18 =
19 8 9th 95% All Ist-lv] illusions
20 a 9th 96 % All 2nd-1v1 illusions
21 10 9th 97 % All 3rd-lvl illusions
22 11 th 98 % All 4th-lv1 illusions
23 12 oth 99 % All 5th-lv1 illusions
24 15 9th 100% All é6th-lv1 illusions
25 20 9th 100% All 7th-lv1 illusions
* While unable to speak a language, the character can still communicate
by grunts and gestures.
why you need to know a given language).
Furthermore, your DM can limit your language selection based on his campaign. It is
perfectly fair to rule that your fighter from
the Frozen Wastes hasn't the tongues of the
Southlands, simply because he has never
met anyone who has been to the Southlands.
If the DM allows characters to have proficiencies, this column also indicates the
number of extra proficiency slots the
character gains due to his Intelligence.
These extra proficiency slots can be used
however the player desires. The character never needs to spend any proficiency
slots to speak his native language.
Spell Level lists the highest level of spells
that can be cast by a wizard with this Intelligence.
Chance to Learn Spell is the percentage
probability that a wizard can learn a particular spell. A check is made as the wizard comes
across new spells, not as he advances in level.
To make the check, the wizard character must
have access to a spell book containing the
spell. If the player rolls the listed percentage or
less, his character can learn the spell and copy
it into his own spell book. If the wizard fails
the roll, he cannot check that spell again until
he advances to the next level (provided he still
has access to the spell).Ability Scores (Wisdom)
Maximum Number of
Spells per Level
Optional Rule)
This number indicates the maximum number of spells a wizard can know from any particular spell
level. Once a wizard has learned the maxisce al rapper
level, he cannot add any more
Sells thet lnvel to he opall Book (urhes
the optional spell research system is used).
Once a spell is learned, it cannot be unlearned and replaced by a new spell.
For example, Delsenora the mage has an
Intelligence of 14. She currently knows
seven 3rd-level spells. During an adventure, she finds a musty old spell book on
the shelves of a dank, forgotten library.
Blowing away the dust, she sees a 3rd-level
spell she has never seen before! Excited,
she sits down and carefully studies the arcane notes, Her chance to learn the spell is
60 percent. Rolling the dice, Delsenora’s
player rolls a 37. She understands the curious instructions and can copy them into
her awn spell book. When she is finished,
she has eight 3rd-level spells, only one
away from her maximum number. If the
die roll had been greater than 60, or she already had nine 3rd-level spells in her spell
book, or the spell had been greater than
7th level (the maximum level her Intelligence allows her to learn), she could not
have added it to her collection.
Spell Immunity is gained by those with
exceptionally high Intelligence scores.
Those with the immunity notice some inconsistency or inexactness in the illusion or
phantasm, automatically allowing them to
make their saving throws. All benefits are
cumulative, thusa character with a 20 Intelligence is not fooled by 1st- or 2nd-level illusion spells.
Wisdom
Wisdom (Wis) describes a composite of the character's enlightenment, judgment, guile, willpower,
common sense, and intuition. It can affect
the character's resistance to magical attack.
It is the prime requisite of priests; those with
a Wisdom score of 16 or higher gain a 10
percent bonus to experience points earned.
Clerics, druids, and other priests with Wisdom scores of 13 or higher also gain bonus
spells over and above the number they are
normally allowed to use.
Magical Defense Adjustment listed on
Table 5 applies to saving throws against
magical spells that attack the mind: beguiling, charm, fear, hypnosis, illusions, possession, suggestion, etc. These bonuses and
penalties are applied automatically, without
any conscious effort from the character.
Bonus Spells indicates the number of additional spells a priest (and only a priest) is
entitled to because of his extreme Wisdom.
Note that these spells are available only
when the priest is entitled to spells of the appropriate level. Bonus spells are cumulative, so a priest with a Wisdom of 15 is
entitled to two Ist-level bonus spells and
one 2nd-level bonus spell.
Chance of Spell Failure states the percentage chance that any particular spell fails
when cast. Priests with low Wisdom scores
run the risk of having their spells fizzle. Roll
percentile dice every time the priest casts a
spell; if the number rolled is less than or
equal to the listed chance for spell failure,
the spell is expended with absolutely no effect whatsoever. Note that priests with Wisdom scores of 13 or higher don’t need to
worry about their spells failing.
Spell Immunity gives those extremely wise
characters complete protection from certain
spells, spell-like abilities, and magical items as
listed. These immunities are cumulative, so
that a character with a Wisdom of 23 is immune to all listed spells up to and including
those listed on the 23 Wisdom row.
Table 5: WISDOM
Magical Chance
Ability Defense Bonus of Spell
Score Adjustment Spells Failure
1 6 _ 80%
2 -4 — 60%
3 =3 — 50%
£ 72 — 45% .
5 red — 40%
6 =f — 35%
7 I] — 30%
8 0 = 23% —
9 0 0-20
10 0 0 15%
1211 00 0 0 10% 5%
aa = a) Ist 0%
14 0 Ist 0%
15 +1 2nd 0%
16 +2 and 0%
17 +3 3rd- (0%
18 +4 4th 0%
19 +4 Ist, 4th = 0%
20 +4 2nd,4th 0%
21 +4 3rd, Sth 0% ear
22 +4 4th, 5th 0% Charm monster, Confusion, Emotion, Fumble, Suggestion
23 +4 Sth, 5th 0% Chaos, Feeblemind, Hold monster, Magic jar, Quest
24 +4 6th, 6th 0% Geas, Mass suggestion, Rod of rulership
25 +4 6th, 7th 0% Antipathy/sympathy, Death spell, Mass charm
Chestanes ber of mercenary soldiers, men-at-arms,
The Charisma (Cha)
score measures a character's persuasiveness,
personal magnetism, and ability to lead. It is
not a reflection of physical attractiveness, although attractiveness certainly plays a role.
It is important to all characters, but especially to those who must deal with nonplayer characters (NPCs), mercenary
hirelings, retainers, and intelligent monsters.
It dictates the total number of henchmen a
character can retain and affects the loyalty of
henchmen, hirelings, and retainers.
Maximum Number of Henchmen states
the number of non-player characters who
will serve as permanent retainers of the
player character. It does not affect the num-
17
servitors, or other persons in the pay of the
character.
Loyalty Base shows the subtraction from
or addition to the henchmen’s and other servitors’ loyalty scores (in the DMG). This is
crucial during battles, when morale becomes important.
Reaction Adjustment indicates the penalty or bonus due to the character because
of Charisma when dealing with non-player
characters and intelligent creatures. For example, Rath encounters a centaur, an intelligent creature. Rath’s Charisma is only 6, so
he is starting off with one strike against him.
He probably should try to overcome this
slight handicap by making generous offers
of gifts or information.Ability Scores (What the Numbers Mean)
Table 6: CHARISMA
Maximum
Ability 4 of Loyalty Reaction
Score Henchmen Base Adjustment
1 0 -8 7
2 1 -7 6
3 1 6 -5
4 1 5 4
5 2 + 3
6 2 <3 -2
7 3 “2 -1
8 3 -1 0
9 4 0 0
10 4 0 0
1] 4 0 0
12 5 0 0
13 5 0 +1
14 6 +] +2
15 7 +3 +3
(16 Pa+ +4 +5
17 10 +6 +6
18 15 +8 +7
19 20 +10 +8
20 25 +12 +9
2 30 +14 +10
22 35 +16 +11
23 40 +18 +12
24 45 +20 +13
25 50 +20 +14
Optional Racial Adjustment. If your
DM is using this rule, your character's
ealing with beings of different races.
(page 20), after the different player character races have been explained.
What the Numbers Mean
Now that you have
finished creating the ability scores for your
character, stop and take a look at them.
What does all this mean?
Suppose you decide to name your character “Rath” and you rolled the following ability scores for him:
Strength 8
Dexterity 14
Constitution 13
Intelligence 13
Wisdom 7
Charisma 6
Rath has strengths and weaknesses, but it
is up to you to interpret what the numbers
mean. Here are just two different ways these
numbers could be interpreted.
1) Although Rath is in good health (Con
13), he’s not very strong (Str 8) because he's
just plain lazy —he never wanted to exercise
as a youth and now it's too late. His low
Wisdom and Charisma scores (7, 6) show
that he lacks the common sense to apply
himself properly and projects a slothful,
“I'm not going to bother” attitude (which
tends to irritate others). Fortunately, Rath’s
natural wit (Int 13) and Dexterity (14) keep
him from being a total loss.
Thus you might play Rath as an irritating, smart-alecky twerp forever ducking
just out of range of those who want to
squash him.
2) Rath has several good points—he has
studied hard (Int 13) and practiced his manual skills (Dex 14). Unfortunately, his
Strength is low (8) from a lack of exercise
(all those hours spent reading books).
Despite that, Rath’s health is still good (Con
13). His low Wisdom and Charisma (7, 6)
are a result of his lack of contact and
involvement with people outside the realm
of academics.
Looking at the scores this way, you could
play Rath as a kindly, naive, and shy professorial type who's a good tinkerer, always
fiddling with new ideas and inventions.
Obviously, Rath’s ability scores (often
called “stats”) are not the greatest in the
world. Yet it is possible to turn these “disappointing” stats into a character who is both
interesting and fun to play. Too often players become obsessed with “good” stats.
These players immediately give up on a
character if he doesn’t have a majority of
above-average scores. There are even those
who feel a character is hopeless if he does
not have at least one ability of 17 or higher!
Needless to say, these players would never
consider playing a character with an ability
score f 6 or 7.
In truth, Rath’s survivability has a lot less
to do with his ability scores than with your
desire to role-play him. If you give up on
him, of course he won't survive! But if you
take an interest in the character and roleplay him well, then even a character with
the lowest possible scores can present a fun,
challenging, and all-around exciting time.
Does he have a Charisma of 57 Why? Maybe he's got an ugly scar. His table manners
could be atrocious. He might mean well but
always manage to say the wrong thing at the
time. He could be bluntly honest to
the point of rudeness, something not likely
to endear him to most people. His Dexterity
is a 3? Why? Is he naturally clumsy or blind
as a bat?
Don't give up on a character just because
he has a low score. Instead, view it as an
opportunity to role-play, to create a unique
and entertaining personality in the game.
Not only will you have fun creating that
personality, but other players and the DM
will have fun reacting to him.

	\end{multicols}
\end{document}