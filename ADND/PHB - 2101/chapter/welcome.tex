\documentclass[../main.tex]{subfiles}
\fancyhead{}
\chead{\textcolor{white}{\changefont{Welcome to the Second Edition AD\&D\textsuperscript{\textregistered}\ Game}}}

\begin{document}
	\footnotesize
	\begin{multicols}{3}
	You are reading the key to the most exciting hobby in the world--role-playing games.
	\indent{These first few pages will introduce you to the second edition of the most successful role-playing game ever published. If you are a novice role-player, stop right here and read the section labeled \textit{The Real Basics} (on the next page). When you understand what role-playing and the AD\&D\textsuperscript{\textregistered}\ game are all about, come back to this point and read the rest of the introduction. If you art' an experienced role-player, skip \textit{The Real Basics}.}
	\section*{\textcolor{dndblue}{Why a Second Edition?}}
	\setlength\parindent{90pt}{Before answering
	that question, let's define what the second
	edition of the AD\&D game is and is not.}
	\setlength\parindent{10pt}\\
	\indent{This second edition of the AD\&D game is a lot different from the first edition. The presentation of the game has been cleaned up. The rules are reorganized, clarified. and streamlined. Where necessary, things that didn't work have been fixed. Things that did work haven't been changed.}\\
	\indent{The second edition of the AD\&D game is not a statement of what anyone person thinks the game should be. It is the result of more than three years of discussion, thought, consultation, review, and play-testing.}\\
	\indent{Now to the question of "Why a second edition?" The AD\&D game evolved over the course of 16 years. During that time, the game grew tremendously through play. Changes and improvements (and a few mistakes) were made. These were published in subsequent volumes. By 1988, the game consisted of 11 hardcover rule books. It was physically and Intellectually unwieldy (but still a lot of fun). The time was right to reorganize and recombine all that information into a manageable package. That package is the second edition.}
	\section*{\textcolor{dndblue}{How the Rule Books\\are Organized}}
	\setlength\parindent{90pt}{The AD\&D game rule books are intended primarily as reference books. They are designed so any specific rule can be found quickly and easily during a game.}\\
	\setlength\parindent{10pt}
	\indent{Everything a player needs to know is in the \textit{Player's Handbook}. That's not to say that all the rules are in this book. But every rule that a player needs to know in order to play the game is in this book.}\\
	\indent{A few rules have been reserved for the Dungeon Master's Guide (DMG). These either cover situations that very seldom arise or give the Dungeon Master (DM) infonnation that players should not have beforehand . Everything else in the DMG is information that only the Dungeon Master needs. If the DM feels that players need to}
	\columnbreak
	
	
	\noindent{know something that is explained in the DMG, he will tell them.}\\
	\indent{Like the DMG, the Monstrous Compendium is the province of the DM. This gives complete and detailed information about the monsters, people, and other creatures inhabiting the AD\&D world. Some DMs don't mind if players read this information, but the game is more fun if players don't know everything about their foes--it heightens the sense of discovery and danger of the unknown.}
	\section*{\textcolor{dndblue}{Learning the Game}}
	\setlength\parindent{90pt}{If you have played the AD\&D game before. you know almost everything you need to play the second edition. We advise you to read the entire Player's Handbook, but the biggest changes are in these chapters: Character Classes, Combat, and Experience. Be sure to read at least those three chapters before sitting down to play,}\\
	\setlength\parindent{10pt}
	\indent{If you come to a term you do not understand, look for it in the Glossary, which
	begins on page 11.}\\
	\indent{If you have never played the AD\&D game before, the best way to leam to play the game is to find a group of experienced players and join them . They can get you immediately into the game and explain things as you need to know them. You don't need to read anything beforehand . In fact, it's best if you can play the game for several hours with experienced players before reading any of the rules. One of the truly marvelous things about a role-playing game is that even though the concept is difficult to explain, It is simple to demonstrate.}\\
	\indent{if none of your friends are involved in a game, the best place to find experienced players is through your local hobby store. Role-playing and general gaming dubs are common and are always eager to accept new members. Many hobby stores offer a bulletin board through which DMs can advertise for new players and new players can ask for information about new or ongoing games, If there is no hobby store in your area, check at the local library Or school.}\\
	\indent{If you can't find anyone else who knows the AO\&D game, you can teach yourself. Read the Player's Handbook and create some characters. Try to create a variety of character classes. Then pick up a prepackaged adventure module for low-level characters, round up two or three friends, and dive into it . You probably will make lots of mistakes and wonder constantly whether you are doing everything wrong. Even if you are, don', worry abou t it. The AD\&D game is big. but eventually you'll bring it under control.}\\
	\columnbreak
	\section*{\textcolor{dndblue}{The Second Edition\\AD\&D Game Line}}
	\setlength\parindent{90pt}{Quite a few books and other products are published for the AD\&D game. As a player, you need only one of them-this book. Every player and DM should have a copy of the Player's Handbook, Everything else is either optional or intended for the Dungeon Master.}\\
	\setlength\parindent{10pt}
	\indent{The Dungeon Master's Guide is essential for the DM and it is for the DM only. Players who are not themselves DMs have no cause to read the DMG.}\\
	\indent{The Monstrous Compendium is not one, but several products . The book can be expanded whenever new compendiums are released. The first pack of monsters is essential to the game. It includes the most commonly encountered monsters, mythical beasts, and legendary creatures. Additional packs expand on these monsters and give the game more variety. Specialty compendiums- those for Greyhawk, the Forgotten Realms. Kara-Tur, etc.-are highly recommended for DMs who play in those settings.}\\
	\indent{Expanded character class books--The Complete fighter, The Complete Thief, etc,--provide a lot more detail on these character classes than does the Player's Handbook. These books are entirely optional. They are for those players who really want a world of choice for their characters.}\\
	\indent{Adventure modules contain complete game adventures. These are especially useful for DMs who aren't sure how to create their own adventures and for DMs who need an adventure quickly and don't have time to write one of their own.}
	\section*{\textcolor{dndblue}{A Note About Pronouns}}
	\setlength\parindent{90pt}{The male pronoun (he, him, his) is used exclusively throughout the second edition of the AD\&D game rules. We hope this won't be construed by anyone to be an attempt to exclude females from the game or imply their exclusion . Centuries of use have neutered the male pronoun. In written material it Is clear, concise, and familiar. Nothing else is.}
	\section*{\textcolor{dndblue}{Creating a Character}}
	\setlength\parindent{90pt}{To create a character to play in the AD\&D game, proceed, in order, through Chapters 1 th rough 6, (Chapter 5 is optional). These chapters will tell you how to generate your character's ability scores. race, and class, decide on his alignment. pick proficiencies, and buy equipment. Once you have done all this, your character is ready for adventure!}
	\pagebreak
\end{multicols}
\end{document}